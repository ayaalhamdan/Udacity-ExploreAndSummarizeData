\documentclass[]{article}
\usepackage{lmodern}
\usepackage{amssymb,amsmath}
\usepackage{ifxetex,ifluatex}
\usepackage{fixltx2e} % provides \textsubscript
\ifnum 0\ifxetex 1\fi\ifluatex 1\fi=0 % if pdftex
  \usepackage[T1]{fontenc}
  \usepackage[utf8]{inputenc}
\else % if luatex or xelatex
  \ifxetex
    \usepackage{mathspec}
  \else
    \usepackage{fontspec}
  \fi
  \defaultfontfeatures{Ligatures=TeX,Scale=MatchLowercase}
\fi
% use upquote if available, for straight quotes in verbatim environments
\IfFileExists{upquote.sty}{\usepackage{upquote}}{}
% use microtype if available
\IfFileExists{microtype.sty}{%
\usepackage{microtype}
\UseMicrotypeSet[protrusion]{basicmath} % disable protrusion for tt fonts
}{}
\usepackage[margin=1in]{geometry}
\usepackage{hyperref}
\hypersetup{unicode=true,
            pdfborder={0 0 0},
            breaklinks=true}
\urlstyle{same}  % don't use monospace font for urls
\usepackage{graphicx,grffile}
\makeatletter
\def\maxwidth{\ifdim\Gin@nat@width>\linewidth\linewidth\else\Gin@nat@width\fi}
\def\maxheight{\ifdim\Gin@nat@height>\textheight\textheight\else\Gin@nat@height\fi}
\makeatother
% Scale images if necessary, so that they will not overflow the page
% margins by default, and it is still possible to overwrite the defaults
% using explicit options in \includegraphics[width, height, ...]{}
\setkeys{Gin}{width=\maxwidth,height=\maxheight,keepaspectratio}
\IfFileExists{parskip.sty}{%
\usepackage{parskip}
}{% else
\setlength{\parindent}{0pt}
\setlength{\parskip}{6pt plus 2pt minus 1pt}
}
\setlength{\emergencystretch}{3em}  % prevent overfull lines
\providecommand{\tightlist}{%
  \setlength{\itemsep}{0pt}\setlength{\parskip}{0pt}}
\setcounter{secnumdepth}{0}
% Redefines (sub)paragraphs to behave more like sections
\ifx\paragraph\undefined\else
\let\oldparagraph\paragraph
\renewcommand{\paragraph}[1]{\oldparagraph{#1}\mbox{}}
\fi
\ifx\subparagraph\undefined\else
\let\oldsubparagraph\subparagraph
\renewcommand{\subparagraph}[1]{\oldsubparagraph{#1}\mbox{}}
\fi

%%% Use protect on footnotes to avoid problems with footnotes in titles
\let\rmarkdownfootnote\footnote%
\def\footnote{\protect\rmarkdownfootnote}

%%% Change title format to be more compact
\usepackage{titling}

% Create subtitle command for use in maketitle
\newcommand{\subtitle}[1]{
  \posttitle{
    \begin{center}\large#1\end{center}
    }
}

\setlength{\droptitle}{-2em}

  \title{}
    \pretitle{\vspace{\droptitle}}
  \posttitle{}
    \author{}
    \preauthor{}\postauthor{}
    \date{}
    \predate{}\postdate{}
  

\begin{document}

\section{Explore and Summarize Data/Red Wine Quality by Ayah
AlHamdan}\label{explore-and-summarize-datared-wine-quality-by-ayah-alhamdan}

\section{Introduction}\label{introduction}

In this project, I will use the data of Red Wine Quality to perform
exploratory data analysis using R to know what influences the quality of
red wines.

\section{Data Overview}\label{data-overview}

First, we'll have a look at the data.

\begin{verbatim}
## Observations: 1,599
## Variables: 13
## $ X                    <int> 1, 2, 3, 4, 5, 6, 7, 8, 9, 10, 11, 12, 13...
## $ fixed.acidity        <dbl> 7.4, 7.8, 7.8, 11.2, 7.4, 7.4, 7.9, 7.3, ...
## $ volatile.acidity     <dbl> 0.700, 0.880, 0.760, 0.280, 0.700, 0.660,...
## $ citric.acid          <dbl> 0.00, 0.00, 0.04, 0.56, 0.00, 0.00, 0.06,...
## $ residual.sugar       <dbl> 1.9, 2.6, 2.3, 1.9, 1.9, 1.8, 1.6, 1.2, 2...
## $ chlorides            <dbl> 0.076, 0.098, 0.092, 0.075, 0.076, 0.075,...
## $ free.sulfur.dioxide  <dbl> 11, 25, 15, 17, 11, 13, 15, 15, 9, 17, 15...
## $ total.sulfur.dioxide <dbl> 34, 67, 54, 60, 34, 40, 59, 21, 18, 102, ...
## $ density              <dbl> 0.9978, 0.9968, 0.9970, 0.9980, 0.9978, 0...
## $ pH                   <dbl> 3.51, 3.20, 3.26, 3.16, 3.51, 3.51, 3.30,...
## $ sulphates            <dbl> 0.56, 0.68, 0.65, 0.58, 0.56, 0.56, 0.46,...
## $ alcohol              <dbl> 9.4, 9.8, 9.8, 9.8, 9.4, 9.4, 9.4, 10.0, ...
## $ quality              <int> 5, 5, 5, 6, 5, 5, 5, 7, 7, 5, 5, 5, 5, 5,...
\end{verbatim}

There are 1599 observations and 13 variables. The X variable is an index
for each observation in the dataset, while the other variables are
chemical properties and the quality of the wine.

\section{Univariate Plots Section}\label{univariate-plots-section}

\begin{quote}
\textbf{Tip}: In this section, you should perform some preliminary
exploration of your dataset. Run some summaries of the data and create
univariate plots to understand the structure of the individual variables
in your dataset. Don't forget to add a comment after each plot or
closely-related group of plots! There should be multiple code chunks and
text sections; the first one below is just to help you get started.
\end{quote}

\section{Univariate Analysis}\label{univariate-analysis}

\begin{quote}
\textbf{Tip}: Now that you've completed your univariate explorations,
it's time to reflect on and summarize what you've found. Use the
questions below to help you gather your observations and add your own if
you have other thoughts!
\end{quote}

\subsubsection{What is the structure of your
dataset?}\label{what-is-the-structure-of-your-dataset}

\subsubsection{What is/are the main feature(s) of interest in your
dataset?}\label{what-isare-the-main-features-of-interest-in-your-dataset}

\subsubsection{\texorpdfstring{What other features in the dataset do you
think will help support your\\
investigation into your feature(s) of
interest?}{What other features in the dataset do you think will help support your investigation into your feature(s) of interest?}}\label{what-other-features-in-the-dataset-do-you-think-will-help-support-your-investigation-into-your-features-of-interest}

\subsubsection{Did you create any new variables from existing variables
in the
dataset?}\label{did-you-create-any-new-variables-from-existing-variables-in-the-dataset}

\subsubsection{\texorpdfstring{Of the features you investigated, were
there any unusual distributions?\\
Did you perform any operations on the data to tidy, adjust, or change
the form\\
of the data? If so, why did you do
this?}{Of the features you investigated, were there any unusual distributions? Did you perform any operations on the data to tidy, adjust, or change the form of the data? If so, why did you do this?}}\label{of-the-features-you-investigated-were-there-any-unusual-distributions-did-you-perform-any-operations-on-the-data-to-tidy-adjust-or-change-the-form-of-the-data-if-so-why-did-you-do-this}

\section{Bivariate Plots Section}\label{bivariate-plots-section}

\begin{quote}
\textbf{Tip}: Based on what you saw in the univariate plots, what
relationships between variables might be interesting to look at in this
section? Don't limit yourself to relationships between a main output
feature and one of the supporting variables. Try to look at
relationships between supporting variables as well.
\end{quote}

\section{Bivariate Analysis}\label{bivariate-analysis}

\begin{quote}
\textbf{Tip}: As before, summarize what you found in your bivariate
explorations here. Use the questions below to guide your discussion.
\end{quote}

\subsubsection{\texorpdfstring{Talk about some of the relationships you
observed in this part of the\\
investigation. How did the feature(s) of interest vary with other
features in\\
the
dataset?}{Talk about some of the relationships you observed in this part of the investigation. How did the feature(s) of interest vary with other features in the dataset?}}\label{talk-about-some-of-the-relationships-you-observed-in-this-part-of-the-investigation.-how-did-the-features-of-interest-vary-with-other-features-in-the-dataset}

\subsubsection{\texorpdfstring{Did you observe any interesting
relationships between the other features\\
(not the main feature(s) of
interest)?}{Did you observe any interesting relationships between the other features (not the main feature(s) of interest)?}}\label{did-you-observe-any-interesting-relationships-between-the-other-features-not-the-main-features-of-interest}

\subsubsection{What was the strongest relationship you
found?}\label{what-was-the-strongest-relationship-you-found}

\section{Multivariate Plots Section}\label{multivariate-plots-section}

\begin{quote}
\textbf{Tip}: Now it's time to put everything together. Based on what
you found in the bivariate plots section, create a few multivariate
plots to investigate more complex interactions between variables. Make
sure that the plots that you create here are justified by the plots you
explored in the previous section. If you plan on creating any
mathematical models, this is the section where you will do that.
\end{quote}

\section{Multivariate Analysis}\label{multivariate-analysis}

\subsubsection{\texorpdfstring{Talk about some of the relationships you
observed in this part of the\\
investigation. Were there features that strengthened each other in terms
of\\
looking at your feature(s) of
interest?}{Talk about some of the relationships you observed in this part of the investigation. Were there features that strengthened each other in terms of looking at your feature(s) of interest?}}\label{talk-about-some-of-the-relationships-you-observed-in-this-part-of-the-investigation.-were-there-features-that-strengthened-each-other-in-terms-of-looking-at-your-features-of-interest}

\subsubsection{Were there any interesting or surprising interactions
between
features?}\label{were-there-any-interesting-or-surprising-interactions-between-features}

\subsubsection{\texorpdfstring{OPTIONAL: Did you create any models with
your dataset? Discuss the\\
strengths and limitations of your
model.}{OPTIONAL: Did you create any models with your dataset? Discuss the strengths and limitations of your model.}}\label{optional-did-you-create-any-models-with-your-dataset-discuss-the-strengths-and-limitations-of-your-model.}

\begin{center}\rule{0.5\linewidth}{\linethickness}\end{center}

\section{Final Plots and Summary}\label{final-plots-and-summary}

\begin{quote}
\textbf{Tip}: You've done a lot of exploration and have built up an
understanding of the structure of and relationships between the
variables in your dataset. Here, you will select three plots from all of
your previous exploration to present here as a summary of some of your
most interesting findings. Make sure that you have refined your selected
plots for good titling, axis labels (with units), and good aesthetic
choices (e.g.~color, transparency). After each plot, make sure you
justify why you chose each plot by describing what it shows.
\end{quote}

\subsubsection{Plot One}\label{plot-one}

\subsubsection{Description One}\label{description-one}

\subsubsection{Plot Two}\label{plot-two}

\subsubsection{Description Two}\label{description-two}

\subsubsection{Plot Three}\label{plot-three}

\subsubsection{Description Three}\label{description-three}

\begin{center}\rule{0.5\linewidth}{\linethickness}\end{center}

\section{Reflection}\label{reflection}

\begin{quote}
\textbf{Tip}: Here's the final step! Reflect on the exploration you
performed and the insights you found. What were some of the struggles
that you went through? What went well? What was surprising? Make sure
you include an insight into future work that could be done with the
dataset.
\end{quote}

\begin{quote}
\textbf{Tip}: Don't forget to remove this, and the other \textbf{Tip}
sections before saving your final work and knitting the final report!
\end{quote}


\end{document}
